\documentclass{report}

\begin{document}
\title{Gear Prototype 1}
\author{GnomeLogic}
\date{June 2012}
\maketitle

\chapter{Character Creation}
In character creation, you make five choices to determine your character's abilities: faction, sub-faction, profession,
role and belief.

\section{Faction}
There are five options for your faction: Gnome, Knight, Undead, Fae or Goblin.

\subsection{Gnome}
The Gnomes are a race of short, eccentric inventors who value ingenuity above all else. They believe that technology can
solve any problem and thus whenever they come across a barrier to their goals, they invent a device to remove it.

\subsection{Knight}
The Knights are elemental beings composed of armour who live in a feudal society. They are born into one of three houses,
the houses of earth, fire and air. They believe in strength through purity, such as of bloodline and thought.

\subsection{Undead}
The Undead are converts from the other four factions, who give up their life and individuality to enter Death's service and
in return are granted eternal unlife. Their existence is one dominated by duty. They believe that Death is the highest
power and know that in time all will fall to it.

\subsection{Fae}
The Fae are forest dwelling magi, who wield nature magic aligned with each of the four seasons. Each Fae has an affinity
for a particular season in which their magic is most powerful and they have the greatest influence over the communal mind
known as the Council. They believe in the power of secrets and the cyclical nature of reality.

\subsection{Goblin}
The Goblins are nomadic scavengers who live in tribes that roam the desert, following the patterns they perceive in the
shifting sands. They believe that every object, force and idea in the world has a spirit to which it is intrinsically tied,
and because of this, have no concept of ownership.

\section{Sub-faction}

\section{Profession}

\section{Role}

\section{Belief}

\chapter{Playing the Game}
\section{Checks}
Each player has a single suit from a standard deck of playing cards which they shuffle and place face down within easy
reach. When a player attempts to do something that the GM deems has a chance of failure, they must make a check by drawing
a card from their pile and place it face up in a private discard pile next to their deck. The card they draw determines
whether they succeed or fail, based on the difficulty level the GM assigns to the action. There are five possible
difficulty levels that can be assigned:
\begin{itemize}
  \item Impossible: no card is drawn and the check automatically fails
  \item Difficult: succeed if a king, queen or jack is drawn, fail otherwise
  \item Medium: succeed if an eight or higher is drawn, fail otherwise
  \item Easy: fail if a one, two or three is drawn, succeed otherwise
  \item Trivial: no card is drawn and the check automatically succeeds
\end{itemize}
% Do we want Impossible and Trivial, and if so do we want them to have a critical chance?

If an ace is drawn, rather than immediately determining success or failure, all thirteen cards are reshuffled and placed
face down and a new card is drawn. This card then indicates success or failure based on the difficulty category, but the
success or failure is considered critical, that is more significant than it would normally be. If the new card is an ace,
immediately draw a third card, indicating supercritical success or failure, before shuffling the cards again. It is up to
the GM to decide exactly what critical or supercritical success or failure means for a particular action.

\section{Challenges}

\end{document}
