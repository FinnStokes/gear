\documentclass{report}

\begin{document}
\title{Gear Prototype 1}
\author{GnomeLogic}
\date{June 2012}
\maketitle

\chapter{Character Creation}

\chapter{Playing the Game}
\section{Checks}
Each player has a single suit from a standard deck of playing cards which they shuffle and place face down within easy
reach. When a player attempts to do something that the GM deems has a chance of failure, they must make a check by drawing
a card from their pile and place it face up in a private discard pile next to their deck. The card they draw determines
whether they succeed or fail, based on the difficulty level the GM assigns to the action. There are five possible
difficulty levels that can be assigned:
\begin{itemize}
  \item Impossible: no card is drawn and the check automatically fails
  \item Difficult: succeed if a king, queen or jack is drawn, fail otherwise
  \item Medium: succeed if an eight or higher is drawn, fail otherwise
  \item Easy: fail if a one, two or three is drawn, succeed otherwise
  \item Trivial: no card is drawm and the check automatically succeeds
\end{itemize}
% Do we want Impossible and Trivial, and if so do we want them to have a critical chance?

If an ace is drawn, rather than immediately determining success or failure, all thirteen cards are reshuffled and placed
face down and a new card is drawn. This card then indicates sucess or failure based on the difficulty category, but the
sucess or failure is considered critical, that is more significant than it would normally be. If the new card is an ace,
immediately draw a third card, indicating supercritical success or failure, before shuffling the cards again. It is up to
the GM to decide exactly what critical or supercritical success or failure means for a particular action.
\section{Challenges}
\end{document}
